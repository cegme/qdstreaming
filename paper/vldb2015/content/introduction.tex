
\section{Introduction}

% What is the problem
The storage of user generated content within systems has introduced 
vast amounts of data.
To correctly process this data must be cleaned. 
Entity resolution is an important part of the ubiquitous cleaning task.
Entity Resolution (ER) is the problem of resolving records in
a data set that correspond the same real world entity.

% Why is the problem hard
Entity resolution is a notoriously computationally difficult problem.
Several efforts in different domains have made outstanding progress~\cite{}.
The main issues still affecting runtimes of ER systems are
twofold, first, the computation of large entities and second, excessive
computation spent resolving unambiguous entities.
Optimization that touches these critical portions is wholly understudied.
We argue that compression and approximation 
techniques can efficiently decrease the runtimes of traditional ER systems.

There is not one size fits all techniques even inside sampling algorithms \cite{sculley2006compression}.

% What are the technical challenges and validation
Some recently, researchers have suggest methods of compressing entities.
Wick et al Heirchical \ldots

Singh et all efficient factoring % http://people.cs.umass.edu/~sameer/files/mcmcmc-emnlp12-ppt.pdf

Each of these methods has drawbacks \ldots

% How we differ
In this paper, we train a multi-class classifier to optimize the decision of
the sampling inference technique to apply.


% Our contributions
We make the following contributions

\begin{itemize}
\item We identify several techniques to speed up sampling past the baseline~\ref{}.
\item We create an optimizer to choose parameters and methods at run time~\ref{}.
\item We empirically evaluate these methods over a large data set~\ref{}.
\end{itemize}




